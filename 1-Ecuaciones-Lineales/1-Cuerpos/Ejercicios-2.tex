\documentclass[10pt,a4paper]{jhwhw}
\usepackage[utf8]{inputenc}
%Paquetes Necesarios
\usepackage{amsmath}
\usepackage{amsfonts}
\usepackage{amssymb}
\usepackage{makeidx}
\usepackage[spanish,es-lcroman]{babel}
\usepackage{titling}
\usepackage{amsthm}
\usepackage{enumerate}
\usepackage{tikz}
\usepackage{latexsym}
\usepackage{cite}
\usepackage{titlesec}
\usepackage{fancybox}
\usepackage{xparse}
%Quitar el identado de todos los parrafos
\setlength{\parindent}{0cm}
%Para agregar el identado en cada item de enumerate o cualquier otro, usar [\hspace{1cm}(a)]

%Comandos de Letras
\newcommand{\R}{\mathbb{R}}
\newcommand{\N}{\mathbb{N}}
\newcommand{\Z}{\mathbb{Z}}
\newcommand{\Q}{\mathbb{Q}}
\newcommand{\C}{\mathbb{C}}

%Marca de agua en el documento
\usepackage{draftwatermark}
\SetWatermarkText{\textsc{\href{https://rull3r.github.io/}{Visitame en MateTips}}} % por defecto Draft 
\SetWatermarkScale{1} % para que cubra toda la página
%\SetWatermarkColor[rgb]{1,0,0} % por defecto gris claro
\SetWatermarkAngle{55} % respecto a la horizontal

\author{Autor: \href{https://www.facebook.com/ruller}{Raúl García}\\Pagina Web: \href{https://rull3r.github.io/}{MateTips}\\Correo: rull3r@hotmail.com}
\date{Venezuela\\ \today \\}
\title{Solucionario \\\href{https://books.google.co.ve/books?id=XPcoPwAACAAJ}{Álgebra Lineal - Hoffman and Kunze}\\}

\makeindex

\begin{document}
	
	\problema{ }\label{pro:2}
	Sea $F$ el cuerpo\index{Cuerpo} de los números\index{Numeros} complejos\index{complejos!Numeros}. ¿Son equivalentes los dos sistemas\index{Sistemas} de ecuaciones lineales\index{ecuaciones lineales!Sistemas} siguiente? Si es así. expresar cada ecuación\index{Ecuacion} de cada sistema como combinación lineal\index{Combinacion lineal} del otro.
	\begin{eqnarray*}
		x_1-x_2=0 & 3x_1+x_2=0\\
		2x_1+x_2=0 & x_1+x_2=0
	\end{eqnarray*}

	\solution
	
	\begin{itemize}
		\item Para la ecuación 1 del sistema 1 hacemos $x_1-x_2=a(3x_1+x_2)+b(x_1+x_2)$
		\begin{align*}
		\begin{aligned}
		1&= 3a+b \\
		-1&=a+b
		\end{aligned}
		\quad
		\Rightarrow
		\quad
		\begin{aligned}
		a&= 1 \\
		b&= -2
		\end{aligned}
		\end{align*}
		
		de modo que $x_1-x_2=1(3x_1+x_2)-2(x_1+x_2)$
		
		\item Para la ecuación 2 del sistema 1 hacemos $2x_1+x_2=a(3x_1+x_2)+b(x_1+x_2)$
		\begin{align*}
		\begin{aligned}
		2&= 3a+b \\
		1&=a+b
		\end{aligned}
		\quad
		\Rightarrow
		\quad
		\begin{aligned}
		a&=\frac{1}{2} \\
		b&= \frac{1}{2}
		\end{aligned}
		\end{align*}
		de modo que $2x_1+x_2=\frac{1}{2}(3x_1+x_2)+\frac{1}{2}(x_1+x_2)$
		
		\item Para la ecuación 1 del sistema 2 hacemos $3x_1+x_2=a(x_1-x_2)+b(2x_1+x_2)$
		\begin{align*}
		\begin{aligned}
		3&= a+2b \\
		1&=-a+b
		\end{aligned}
		\quad
		\Rightarrow
		\quad
		\begin{aligned}
		a&=\frac{1}{3} \\
		b&= \frac{4}{3}
		\end{aligned}
		\end{align*}
		de modo que $3x_1+x_2=\frac{1}{3}(x_1-x_2)+\frac{4}{3}(2x_1+x_2)$
		\item Para la ecuación 2 del sistema 2 hacemos $x_1+x_2=a(x_1-x_2)+b(2x_1+x_2)$
		\begin{align*}
		\begin{aligned}
		1&= a+2b \\
		1&=-a+b
		\end{aligned}
		\quad
		\Rightarrow
		\quad
		\begin{aligned}
		a&=\frac{-1}{3} \\
		b&= \frac{2}{3}
		\end{aligned}
		\end{align*}
		de modo que $x_1+x_2=\frac{-1}{3}(x_1-x_2)+\frac{2}{3}(2x_1+x_2)$
	\end{itemize}
	
	
	
\end{document}